\section{Introduction}
\label{sec:introduction}

Medical image classification represents one of the most critical applications of machine learning, where prediction errors can have severe clinical consequences ranging from delayed treatment to unnecessary interventions.
The multi-label nature of medical diagnosis, particularly in chest radiography, presents unique challenges that distinguish it from standard classification tasks.
A single chest X-ray may reveal multiple co-occurring pathologies such as cardiomegaly alongside pleural effusion, or infiltration patterns associated with pneumonia.
Traditional machine learning approaches typically treat each diagnostic label independently, thereby ignoring the rich structure of dependencies that exists between different medical conditions.

These dependencies are not merely statistical correlations but often reflect underlying pathophysiological mechanisms.
For instance, cardiomegaly (an enlarged heart) frequently causes pleural effusion (fluid accumulation around the lungs) through congestive heart failure mechanisms, while infiltration patterns in lung tissue commonly precede or accompany pneumonia.
A radiologist naturally considers these relationships when interpreting images, asking questions like \("\)given that I see infiltration, should I look more carefully for signs of pneumonia?\("\) This conditional reasoning process suggests that machine learning models would benefit from explicitly modeling such asymmetric dependencies rather than treating each label as an independent prediction problem.

Beyond accurate point predictions, clinical deployment of machine learning systems demands rigorous uncertainty quantification.
Conformal prediction offers an elegant framework for constructing prediction sets with finite-sample coverage guarantees that hold distribution-free, without requiring specific assumptions about the underlying data generating process~\cite{vovk2005algorithmic}.
This property makes conformal prediction particularly attractive for medical applications where distributional assumptions are difficult to verify and model misspecification is common.
However, most existing conformal prediction methods for multi-label problems either treat labels independently or model dependencies through symmetric measures like mutual information, failing to capture the directional nature of many clinical relationships.

In this work, we introduce Co-occurrence Weighted Conformal Sets (CWCS), a novel conformal prediction framework that explicitly leverages asymmetric label dependencies estimated from empirical co-occurrence patterns in training data.
Our method constructs prediction sets by incorporating weighted nonconformity scores that account for how strongly the presence of one pathology predicts the presence of another, while recognizing that this relationship may not be symmetric.
Through comprehensive experiments on the ChestX-ray14 dataset encompassing over 112,000 radiographs from more than 30,000 patients, we evaluate CWCS against three established baseline methods and provide detailed analysis of when and why dependency modeling provides value in practical medical imaging applications.

